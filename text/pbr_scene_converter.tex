\documentclass[cic,tc,english]{iiufrgs}

\usepackage[utf8]{inputenc}   % pacote para acentuação
\usepackage{graphicx}         % pacote para importar figuras
\usepackage{times}            % pacote para usar fonte Adobe Times
% \usepackage{palatino}
% \usepackage{mathptmx}       % p/ usar fonte Adobe Times nas fórmulas
\usepackage[alf,abnt-emphasize=bf]{abntex2cite}	% pacote para usar citações abnt


\title{Um Exemplo de Monografia do Instituto de Informática da UFRGS}
\author{Hagemann}{Luiza de Azambuja}
\advisor[Prof.~Dr.]{Neto}{Manuel Menezes de Oliveira}

\keyword{formatação eletrônica de documentos}
\keyword{\LaTeX}
\keyword{ABNT}
\keyword{UFRGS}

\begin{document}

\maketitle

% dedicatoria
% \clearpage
% \begin{flushright}
%     \mbox{}\vfill
%     {\sffamily\itshape
%       ``If I have seen farther than others,\\
%       it is because I stood on the shoulders of giants.''\\}
%     --- \textsc{Sir~Isaac Newton}
% \end{flushright}

% agradecimentos
%\chapter*{Agradecimentos}
%Agradeço ao \LaTeX\ por não ter vírus de macro\ldots

% resumo na língua do documento
\begin{abstract}
    Este documento é um exemplo de como formatar documentos para o
    Instituto de Informática da UFRGS usando as classes \LaTeX\
    disponibilizadas pelo UTUG\@. Ao mesmo tempo, pode servir de consulta
    para comandos mais genéricos. \emph{O texto do resumo não deve
      conter mais do que 500 palavras.}
\end{abstract}

% resumo na outra língua
% como parametros devem ser passados o titulo e as palavras-chave na outra língua, separadas por vírgulas
\begin{englishabstract}{Using \LaTeX\ to Prepare Documents at II/UFRGS}{Electronic document preparation. \LaTeX. ABNT. UFRGS}
    This document is an example on how to prepare documents at II/UFRGS
    using the \LaTeX\ classes provided by the UTUG\@. At the same time, it
    may serve as a guide for general-purpose commands. \emph{The text in
      the abstract should not contain more than 500~words.}
\end{englishabstract}

% lista de figuras
\listoffigures

% lista de tabelas
\listoftables

% lista de abreviaturas e siglas
\begin{listofabbrv}{SPMD}
    \item[SMP] Symmetric Multi-Processor
    \item[NUMA] Non-Uniform Memory Access
    \item[SIMD] Single Instruction Multiple Data
    \item[SPMD] Single Program Multiple Data
    \item[ABNT] Associação Brasileira de Normas Técnicas
\end{listofabbrv}

% idem para a lista de símbolos
% \begin{listofsymbols}{$\alpha\beta\pi\omega$}
%     \item[$\sum{\frac{a}{b}}$] Somatório do produtório
%     \item[$\alpha\beta\pi\omega$] Fator de inconstância do resultado
% \end{listofsymbols}

% sumario
\tableofcontents

% capitulos
\chapter{Introduction}
No início dos tempos, Donald E. Knuth criou o \TeX. Algum tempo depois, Leslie Lamport criou o \LaTeX. Graças a eles, não somos obrigados a usar o Word nem o LibreOffice.

\chapter{Related Works}
\chapter{Theoretical Foundations}
\chapter{Project}
\chapter{Result Analysis}
\chapter{Conclusion}

% referências
% aqui será usado o environment padrao `thebibliography'; porém, sugere-se
% seriamente o uso de BibTeX e do estilo abnt.bst (veja na página do
% UTUG)

\bibliographystyle{abntex2-alf}
\bibliography{biblio}

\end{document}
