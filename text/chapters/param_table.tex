
\begin{table}
  \begin{center}
      \begin{tabular}{ | L{2.5cm} | L{4cm} | L{4.2cm} | L{2.4cm} | }
      \hline
      {\bf Component} & {\bf ~~~~Type} & {\bf Parameters} & {\bf Others} \\ 
      \hline
      {\bf camera} & environment, orthographic, perspective, realistic & focal distance, fov, lens aperture, near/far clip, shutter open/close & view matrix \\
      \hline
      {\bf sampler} & halton, random, sobol, stratified & samples per pixel, scramble & ~~~~~~- \\
      \hline
      {\bf film} & hdr, ldr & file extension (png, ...), filter, image height, image width & ~~~~~~- \\
      \hline
      {\bf integrator} & bidirectional path tracer, direct lighting, metropolis light transport, path tracer, photon mapping & max depth, number of iterations, number of Markov chains, photon count, photon mapping lookup radius, russian roulette depth & ~~~~~~- \\
      \hline
      {\bf materials} & glass, matte/diffuse, metal, substrate/glossy, translucent, uber & $\eta$, id, IOR, k, kd, ks, reflectance, roughness, transmittance, ... & texture (id, type, params) \\ 
      \hline
      {\bf emitters} & directional, distant, environment mapping, sky, spot, sun & filename, from (origin), intensity, radiance, to (direction) & model matrix \\
      \hline
      {\bf shapes} & mesh (ply/obj) & filename & model matrix, area emitter, \\
      \cline{2-3}
	    & rectangle, disk, triangle mesh, cube, sphere & center, normals, points, radius, uv mapping, ... & unnamed material\\
      \hline

      \end{tabular}
  \end{center}
    \caption{Types, parameters and additional attributes of the components in our canonical scene representation (Figure~\ref{fig:canonicalrep}).} 
    \label{tab:summary}
\end{table}
    