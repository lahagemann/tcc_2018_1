\section{Introduction}

Ever since the development of modern Computer Graphics, one of the goals 
researchers aspired to was being able to synthesize images indistinguishable 
from real photographs. In order to produce physically accurate images, the 
process of image synthesis - also called \textbf{rendering} - simmulates the 
interaction of light with the representation of a three-dimensional scene. 

\textbf{Physically-Based Rendering (PBR)} is a complex process that requires 
thorough knowledge of optics, material properties, geometry and light 
propagation.

\subsection{Physically Based Rendering (PBR)}
Over the years, PBR became quite popular and was widely incorporated into the 
entertainment industries. From movies to videogames, from ads to interior 
design, PBR made it possible for artists to bring their creations and their 
vision one step closer to reality. Today, we can say that many - if not most - 
algorithms used in computer animation, geometric modeling and texturing require 
that their results be passed through some sort of rendering process.

As PBR popularity grew, a brand new market opened up for physically-based (PB) 
renderers. Following the creation of \textit{PBRT} \cite{pbrt} and the publishing of 
\textit{"Physically Based Rendering: From Theory to Implementation"} 
\cite{pbrtBook}, several other research-oriented renderers were created. Among them 
is \textit{Mitsuba} \cite{mitsuba}, one of the renderers chosen for this research, which places 
strong emphasis on experimental rendering techniques.

Following the lead of Pixar's \textit{Renderman}, many 
commercial and performance-oriented renderers appeared on the market. Focused on 
animation techniques and visual effects for movies, these renderers provide 
well-established, stable rendering techniques. These renderers, such as 
\textit{LuxRender} \cite{luxrender} and \textit{Octane}, are 
state-of-the-art renderers used by the animation and gaming industries.
%https://www.blenderguru.com/articles/render-engine-comparison-cycles-vs-giants

Even with different applications, the vast majority of modern PB renderers 
follows the same general guidelines for defining scene directives and world 
descriptions. Scene directives establish parameters such as which integration 
and sampling techniques the renderer must use, the view matrix and other camera 
properties. World descriptions state which objects compose the scene and which 
materials must be used to render them. This ensemble of descriptions is what, in 
PBR, is called a \textbf{scene}.

\subsection{Rendering a Scene}
% - the scene
Starting with \textit{PBRT}, most PB renderers have used a similar, traditional​ 
structure to describe scenes.

These scenes are usually created by a 3D artist, who will use a modeling 
software (such as Blender, 3D Max or Maya) to draw the objects, choose their 
materials and then create the object files. These files will then be instanced 
in the scene file and interpreted by the renderer.

% - stating the problem/motivation: making scenes is hard 
% (reference: http://www.laubwerk.com/home/)
But even with an artist’s expertise, creating scenes is still a complex process. 
For instance, scenes created for building overviews and interior design often 
compile hundreds of 3D models and dozens of customized materials and textures. Each material and texture 
has to be carefully defined, taking into account the renderer's limitations and 
particularities.

After a scene is created and rendered, all the hard work invested by the artist 
is stored, waiting for a possible future use. However, should the artist choose 
to change renderers, the scene file they created so diligently would have to be 
rewritten and/or heavily modified.

Converting a scene file from one renderer format to another is very difficult 
and time consuming. Aside from adapting material and light properties - which 
can be hard since sometimes renderers don't provide the same features -, the 3D 
object formats supported may not be the same. For instance, \textit{Mitsuba} 
supports the Object File Wavefront 3D (.obj) format while \textit{PBRT} does 
not.

Creating free resources in order to help rendering research, Benedikt Bitterli 
\cite{resources16} manually converted 32 scenes for his renderer \textit{Tungsten} 
\cite{tungsten}, later converting them to \textit{PBRT} and \textit{Mitsuba}. 
According to him, \textit{"this process is time intensive (up to several days 
per scene)"}.