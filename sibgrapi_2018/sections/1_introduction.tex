\section{Introduction}
\label{sec:introduction}

Monte Carlo ray tracing is currently the only practical solution for simulating global illumination effects in complex environments.
Due to its high computational cost, several techniques have been introduced to reduce rendering time through improved sampling~\cite{Heck2013, Pilleboue:2015} and reconstruction strategies~\cite{Sen2012, Rousselle2013, Kalantari2015, Bitterli2016}. When developing such new techniques, researchers often implement them on top of existing rendering systems as a way of leveraging available infrastructure to perform functions (\eg, ray-traversal acceleration, ray-primitive intersections, etc.) that are orthogonal to the proposed methods.

Unfortunately, the various rendering systems use proprietary scene description formats. While modeling visually-pleasing scenes requires significant artistic skills, manual conversion between proprietary formats requires knowledge of the specific formats and tend to be extremely time consuming (up to several days per scene~\cite{tungsten}). 
Thus, by selecting a given rendering system one is often constrained to test and demonstrate the proposed techniques on the limited set of test scenes available for that renderer. This apparently simple limitation has profound implications, as it constrains a direct comparison between Monte Carlo (MC) rendering techniques that have been implemented using different rendering systems. In this case, one often has to compare the quality of algorithms using disjoint sets of scenes, which tends to be unsatisfactory.    
    
We present {\it a system for automatic conversion among scene file formats used by Monte Carlo physically-based rendering systems}. 
Our solution significantly expands the repertoire of scenes available for testing, validation, and benchmarking of MC rendering algorithms.  
Currently, our system handles conversions among PBRT v3~\cite{PBRT:v3}, Mitsuba~\cite{mitsuba}, and LuxRender~\cite{luxrender}, which are three of the most popular physically-based renderers (PBR). Our solution (discussed in Section~\ref{sec:systemarch}) consists of {\it importing} any source scene description into a canonical representation, which can then be {\it exported} to other scene formats. By specializing the import and export classes to the various formats, our system can convert among arbitrary scene file formats.     
%
Figure~\ref{fig:teaser} illustrates the use of our system to perform automatic conversion of a scene originally modeled for PBRT v3. The image shown on the left is the original PBRT v3 rendering. The images on the center and on the right show, respectively, the renderings produced by Mitsuba and by LuxRender, from the converted scenes files. Note the correct representation of the scene elements, despite the many differences in the scene representations (\eg, different coordinate systems, (\red{what else?}).

Our work does not introduce a new physically-based rendering technique per se. Instead, it falls in the area of {\it meta-research} systems, which are systems designed to facilitate and improve the research process. Meta-research systems are quite common in computer graphics~\cite{Santos:2018:FBKSD, Ragan-Kelley2012,Buck2004,Mark2003} and computer vision~\cite{MiddleburyStereo, MiddleburyFlow, AlphaMatting, VideoMatting}, where they have led to significant progress in these fields. In a work particularly relevant to us of Santos et al.~\cite{Santos:2018:FBKSD} introduced a framework for developing and benchmarking MC sampling and denoising algorithms. This is achieved by providing an API the decouples the developed techniques from the used rendering system. While it allows a technique to be tested on any rendering system that supports the proposed API, each rendering system is still constrained to work on its limited set of test scenes. Our system is orthogonal to and complements the API described in~\cite{Santos:2018:FBKSD}, lending to full orthogonality, by allowing an MC rendering technique to be tested on any combination of rendering systems and scene file.     

The {\bf contributions} of our work include:
\begin{itemize}
	\item A system for automatic conversion among scene file formats used by Monte Carlo physically-based rendering systems (Section~\ref{sec:systemarch}).
	It enables algorithms implemented on different rendering systems to be tested on similar scene description, giving developers and end users a better assessment of the strengths and limitations of MC rendering techniques;
	\item A mechanism for achieving full orthogonality among MC rendering algorithms, rendering systems, and scene files (Section~\ref{sec:systemarch}). This is achieved in combination with the API provided in~\cite{Santos:2018:FBKSD}. 
\end{itemize}

%
%Ever since the development of modern Computer Graphics, one of the goals 
%researchers aspired to was being able to synthesize images indistinguishable 
%from real photographs. In order to produce physically accurate images, the 
%process of image synthesis - also called \textbf{rendering} - simmulates the 
%interaction of light with the representation of a three-dimensional scene. 
%
%\textbf{Physically-Based Rendering (PBR)} is a complex process that requires 
%thorough knowledge of optics, material properties, geometry and light 
%propagation.
%
%\subsection{Physically Based Rendering (PBR)}
%Over the years, PBR became quite popular and was widely incorporated into the 
%entertainment industries. From movies to videogames, from ads to interior 
%design, PBR made it possible for artists to bring their creations and their 
%vision one step closer to reality. Today, we can say that many - if not most - 
%algorithms used in computer animation, geometric modeling and texturing require 
%that their results be passed through some sort of rendering process.
%
%As PBR popularity grew, a brand new market opened up for physically-based (PB) 
%renderers. Following the creation of \textit{PBRT} \cite{pbrt} and the publishing of 
%\textit{"Physically Based Rendering: From Theory to Implementation"} 
%\cite{pbrtBook}, several other research-oriented renderers were created. Among them 
%is \textit{Mitsuba} \cite{mitsuba}, one of the renderers chosen for this research, which places 
%strong emphasis on experimental rendering techniques.
%
%Following the lead of Pixar's \textit{Renderman}, many 
%commercial and performance-oriented renderers appeared on the market. Focused on 
%animation techniques and visual effects for movies, these renderers provide 
%well-established, stable rendering techniques. These renderers, such as 
%\textit{LuxRender} \cite{luxrender} and \textit{Octane}, are 
%state-of-the-art renderers used by the animation and gaming industries.
%%https://www.blenderguru.com/articles/render-engine-comparison-cycles-vs-giants
%
%Even with different applications, the vast majority of modern PB renderers 
%follows the same general guidelines for defining scene directives and world 
%descriptions. Scene directives establish parameters such as which integration 
%and sampling techniques the renderer must use, the view matrix and other camera 
%properties. World descriptions state which objects compose the scene and which 
%materials must be used to render them. This ensemble of descriptions is what, in 
%PBR, is called a \textbf{scene}.
%
%\subsection{Rendering a Scene}
%% - the scene
%Starting with \textit{PBRT}, most PB renderers have used a similar, traditional​ 
%structure to describe scenes.
%
%These scenes are usually created by a 3D artist, who will use a modeling 
%software (such as Blender, 3D Max or Maya) to draw the objects, choose their 
%materials and then create the object files. These files will then be instanced 
%in the scene file and interpreted by the renderer.
%
%% - stating the problem/motivation: making scenes is hard 
%% (reference: http://www.laubwerk.com/home/)
%But even with an artist’s expertise, creating scenes is still a complex process. 
%For instance, scenes created for building overviews and interior design often 
%compile hundreds of 3D models and dozens of customized materials and textures. Each material and texture 
%has to be carefully defined, taking into account the renderer's limitations and 
%particularities.
%
%After a scene is created and rendered, all the hard work invested by the artist 
%is stored, waiting for a possible future use. However, should the artist choose 
%to change renderers, the scene file they created so diligently would have to be 
%rewritten and/or heavily modified.
%
%Converting a scene file from one renderer format to another is very difficult 
%and time consuming. Aside from adapting material and light properties - which 
%can be hard since sometimes renderers don't provide the same features -, the 3D 
%object formats supported may not be the same. For instance, \textit{Mitsuba} 
%supports the Object File Wavefront 3D (.obj) format while \textit{PBRT} does 
%not.

%Creating free resources in order to help rendering research, Benedikt Bitterli 
%\cite{resources16} manually converted 32 scenes for his renderer \textit{Tungsten} 
%\cite{tungsten}, later converting them to \textit{PBRT} and \textit{Mitsuba}. 
%According to him, \textit{"this process is time intensive (up to several days 
%per scene)"}.