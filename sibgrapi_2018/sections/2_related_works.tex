\section{Related Work}
\label{(sec:related_work)}

To the best of our knowledge, no previous system has performed automatic scene conversion among the major MC rendering systems. 
Bitterli has converted 32 scenes of various complexities and origins from Tungsten to PBRT v3 and Mitsuba using some scripts~\cite{tungsten}. These scripts, however, are specific for conversions from Tungsten to these two renderers and are not publicly available. 

RenderToolbox3 is a MATLAB tool developed for assisting vision research~\cite{rendertoolbox}. It imports a scene containing geometric objects described as COLLADA XML files, and allows one to associate to them reflectance measurements taken from a MATLAB Psycophysics Toolbox~\cite{Brainard1997}. Such reflectance measurements are converted to multispectral reflectance representations compatible to PBRT and Mitsuba. A script then renders the objects with the associated multispectral representations using PBRT or Mitsuba. RenderToolbox3 is a visualization tool for exploring the impact of different reflectance and illuminating properties on human perception.       
%Heasly et al. developed a tool to aid the studying of vision: RenderToobox3 \cite{rendertoolbox} is a system that imports COLLADA XML files into a MATLAB system and uses different rendering techniques in order images as close to reality as possible. Their system uses integrated PBRT and Mitsuba rendering is capable of storing their data structure into their respective scene file formats. 
%As a tool for visualization, 
The system does not import scenes from Mitsuba, PBRT or other renderers, and does convert scene files among rendering systems. 

\subsection{Meta-Research Systems in Computer Graphics}

Several systems have been developed to support research in computer graphics. 
Some well-known examples include Cg~\cite{Mark2003}, Brook~\cite{Buck2004}, and Halide~\cite{Ragan-Kelley2012}.
Cg is a general-purpose programming language designed to support the development of efficient GPU applications, and has stimulated a lot of research efforts in shader-based rendering techniques~\cite{Policarpo2005, Policarpo2006, Wyman2005, Oliveira2007RRT}.  
Brook~\cite{Buck2004} is system for general-purpose computation that allows one to exploit the inherent parallelism of modern GPUs without having to deal with GPU architecture details. These kinds of systems were an inspiration that led to the development of CUDA~\cite{Nickolls2008}.
Halide~\cite{Ragan-Kelley2012} is a system designed to optimize image-processing applications on multiple hardware platforms by separating the algorithm description from its schedule. The system has been recently extended to support differentiable programming for image processing and deep learning~\cite{Li2018-Halide-Diff_Prog}. All these systems focus on generating efficient code while freeing the user from GPU architectural details. All goal, in turn, is to make high-quality scenes availability independent of one's choice of rendering system. 

Santos et al. have recently presented a framework for developing and benchmarking MC sampling and denoising algorithms~\cite{Santos:2018:FBKSD}.
They use an API to decouple algorithms from rendering systems, allowing for the same algorithm to be tested on multiple rendering systems. By doing so, they also increase the set of scenes an algorithm can be tested with. However, in order to use a given test scene, the rendering system for which the scene was created would have to be used as well. Our scene-conversion solution complements the API described in~\cite{Santos:2018:FBKSD}, lending to a desirable full orthogonality among MC algorithms, rendering systems, and scene files.

%With the addition of new physically-based renderers to the market, the number of 
%incompatible file formats began to grow. There were a few initiatives to solve 
%this problem, the greatest one being the proposal of a common file format.
%
%Back in 2004, Arnaud and Barnes created COLLADA \cite{collada}, a XML schema 
%file format that intended to unify representation of digital assets among 
%various graphics software applications. Ever since it became property of the 
%Khronos Group, several companies included a COLLADA module on their 3D modeling 
%softwares or game engines. However, there were few physically-based renderers 
%that adhered to this file format, one of the few being Mitsuba. That might have 
%happened because COLLADA files only include information about the geometry 
%present in the scene - they don't store any information about other rendering 
%options, such as camera positioning or integration techniques.
%
%While working on his renderer, Tungsten \cite{tungsten}, Benedikt Bitterli 
%manually converted 32 scenes of various complexities and origins. He later 
%released his work as an open source set for rendering research. \textcolor{red}{He later 
%exported these manually converted scenes from Tungsten to Mitsuba and PBRT v3 
%using automatic converters}. These converters were not released to the public, 
%though. Bitterli's work was crucial to the development of our system: we used 
%his set of rendering resources to evaluate our system's results.


