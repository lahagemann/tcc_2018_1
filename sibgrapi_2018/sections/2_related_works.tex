\section{Related Work}
\label{(sec:related_work)}

To the best of our knowledge, no previous system has performed automatic scene conversion among the major MC rendering systems. 
Bitterli has converted 32 scenes of various complexities and origins from Tungsten to PBRT v3 and Mitsuba using some scripts~\cite{tungsten}. These scripts, however, are specific for conversions from Tungsten to these two renderers and are not publicly available. 


\subsection{Meta-Research Systems in Computer Graphics}

Several systems have been developed to support research in computer graphics. 
Some well-known examples include Cg~\cite{Mark2003}, Brook~\cite{Buck2004}, and Halide~\cite{Ragan-Kelley2012}.
Cg is a general-purpose programming language designed to support the development of efficient GPU applications, and has stimulated a lot of research efforts in shader-based rendering techniques~\cite{Policarpo2005, Policarpo2006, Wyman2005, Oliveira2007RRT}.  
Brook is system for general-purpose computation that allows one to exploit the inherent parallelism of modern GPUs without having to deal with GPU architecture details. These kinds of systems were an inspiration that led to the development of CUDA~\cite{Nickolls2008}.


%With the addition of new physically-based renderers to the market, the number of 
%incompatible file formats began to grow. There were a few initiatives to solve 
%this problem, the greatest one being the proposal of a common file format.
%
%Back in 2004, Arnaud and Barnes created COLLADA \cite{collada}, a XML schema 
%file format that intended to unify representation of digital assets among 
%various graphics software applications. Ever since it became property of the 
%Khronos Group, several companies included a COLLADA module on their 3D modeling 
%softwares or game engines. However, there were few physically-based renderers 
%that adhered to this file format, one of the few being Mitsuba. That might have 
%happened because COLLADA files only include information about the geometry 
%present in the scene - they don't store any information about other rendering 
%options, such as camera positioning or integration techniques.
%
%While working on his renderer, Tungsten \cite{tungsten}, Benedikt Bitterli 
%manually converted 32 scenes of various complexities and origins. He later 
%released his work as an open source set for rendering research. \textcolor{red}{He later 
%exported these manually converted scenes from Tungsten to Mitsuba and PBRT v3 
%using automatic converters}. These converters were not released to the public, 
%though. Bitterli's work was crucial to the development of our system: we used 
%his set of rendering resources to evaluate our system's results.


