%==========================================
%
% SIBGRAPI 2018 paper
% Example of IEEEtran.cls, adapted for SIBGRAPI 2018 
%
%==========================================

% *** Authors should verify (and, if needed, correct) their LaTeX system  ***
% *** with the testflow diagnostic prior to trusting their LaTeX platform ***
% *** with production work. The IEEE's font choices and paper sizes can   ***
% *** trigger bugs that do not appear when using other class files.       ***                          ***
% The testflow support page is at:
% http://www.michaelshell.org/tex/testflow/

\documentclass[10pt,conference]{IEEEtran}

\usepackage{cite}

\ifCLASSINFOpdf
   \usepackage[pdftex]{graphicx}
   \graphicspath{{figs/}}
   \DeclareGraphicsExtensions{.pdf,.jpeg,.png}
\else
   \usepackage[dvips]{graphicx}
   \graphicspath{{../figs/}}
   \DeclareGraphicsExtensions{.eps}
\fi

\usepackage[cmex10]{amsmath}
\interdisplaylinepenalty=2500
\usepackage{amsthm}
\newtheorem{definition}{Definition}
\usepackage{algorithmic}
\usepackage{array}
\newcolumntype{L}[1]{>{\raggedright\let\newline\\\arraybackslash\hspace{0pt}}m{#1}}
\ifCLASSOPTIONcompsoc
  \usepackage[caption=false,font=normalsize,labelfont=sf,textfont=sf]{subfig}
\else
  \usepackage[caption=false,font=footnotesize]{subfig}
\fi
\usepackage{url}
\hyphenation{op-tical net-works semi-conduc-tor}

\usepackage{xcolor}
\usepackage{subfig}

\newcommand \red[1] {\textcolor{red}{#1}}
\newcommand \green[1] {\textcolor{green}{#1}}
\newcommand \eg {{\it e.g.}}
\newcommand \ie {{\it i.e.}}
\usepackage{caption}



\title{Scene Conversion for Physically-based Renderers}

\newif\iffinal
\finalfalse
%\finaltrue
\newcommand{\jemsid}{51}
\iffinal

\author{\IEEEauthorblockN{Luiza Hagemann}
\IEEEauthorblockA{Instituto de Inform\'{a}tica - UFRGS\\
%Universidade Federal do Rio Grande do Sul\\
Porto Alegre, RS, Brazil\\
lahagemann@inf.ufrgs.br}
\and
\IEEEauthorblockN{Manuel M. Oliveira}
\IEEEauthorblockA{Instituto de Inform\'{a}tica - UFRGS\\
%Universidade Federal do Rio Grande do Sul\\
Porto Alegre, RS, Brazil\\
oliveira@inf.ufrgs.br}}

\else
  \author{SIBGRAPI paper ID: \jemsid \\ }
\fi

\begin{document}
\twocolumn[{%
	\renewcommand\twocolumn[1][]{#1}%
	\maketitle
	\begin{center}
		\centering
	\includegraphics[width=0.2\linewidth]{figs/4_results/coffee/1_from_pbrt.png}
	\includegraphics[width=0.2\linewidth]{figs/4_results/coffee/2_to_mitsuba.png}
	\includegraphics[width=0.2\linewidth]{figs/4_results/coffee/3_to_lux.png}
	\captionof{figure}{Example of automatic scene conversion obtained with our system. \textit{Coffee Maker} rendered with PBRT v3 (left). Rendering produced by Mitsuba (center) and LuxRender (right), using converted scenes (from PBRT v3) for these rendering systems.}
	\label{fig:teaser}
	\end{center}%
}]	
	
%\maketitle

\begin{abstract}
Physically-based rendering systems use proprietary scene description formats. By selecting a given renderer for the development of a new technique, one is often constrained to test and demonstrate it on the limited set of test scenes available for that renderer. This makes it difficult to compare techniques implemented on different rendering systems. 
We present a system for automatic conversion among scene description formats used by physically-based rendering systems. 
It enables algorithms implemented on different renderers to be tested on similar scene, providing better means of assessing their strengths and limitations. 
Our system can be integrated with existing development and benchmarking APIs, providing full orthogonality among algorithms, rendering systems, and scene files.
\end{abstract}

\IEEEpeerreviewmaketitle

\chapter{Introduction}
\label{sec:intro}

%Ever since the development of 
In Computer Graphics, 
%one of the goals 
researchers  have long pursued the goal of synthesizing images indistinguishable 
from real photographs. In order to produce physically accurate images, the 
process of image synthesis - also called \textit{rendering} - has to simulate the 
interaction of light with the representation of a three-dimensional scene. 
%
\textit{Physically-based rendering (PBR)} is a complex process that requires 
thorough knowledge of optics, material properties, geometry and light 
propagation.

\section{Physically-based Rendering}
Physically-based rendering is often implemented using Monte Carlo (MC) Ray Tracing \cite{mcraytracing}, which uses MC 
integration \cite{montecarlo} to estimate the environment illumination function. 
This is performed by tracing (or sampling) the path of several light rays starting from the camera position, simulating the effects obtained from its encounters with virtual objects. 
While able to produce a high degree of realism, this technique also has a very 
high computational cost. This method will be further discussed in Chapter~\ref{sec:theory}.

Over the years, PBR became quite popular and was widely incorporated by the 
entertainment industry. From movies to videogames, from advertisement to interior 
design, PBR made it possible for artists to bring their 
vision one step closer to reality. 
%Today, we can say that many - if not most - algorithms used in computer animation, geometric modeling and texturing require that their results be passed through some sort of rendering process.

As PBR popularity grew, several new renderers were developed.
%a brand new market opened up for physically-based renderers. 
Following the creation of \textit{PBRT} and the publishing of the book 
\textit{"Physically Based Rendering: From Theory to Implementation"} 
\cite{pbrt}, several other research-oriented renderers were created. Among them 
is \textit{Mitsuba} \cite{mitsuba}, one of the renderers chosen for this research.
%, which places strong emphasis on algorithms not yet established in the industry.

Following the lead of Pixar's \textit{Renderman} \cite{renderman}, many 
commercial and performance-oriented renderers appeared on the market. Focused on 
animation techniques and visual effects for movies, these renderers provide 
well-established, stable rendering techniques. These renderers, such as 
\textit{LuxRender} \cite{luxrender} and \textit{Octane} \cite{octane}, 
%are state-of-the-art renderers 
have been extensively used by the animation and gaming industries.
%https://www.blenderguru.com/articles/render-engine-comparison-cycles-vs-giants

Even with different applications, the vast majority of modern physically-based renderers 
follows the same general guidelines for defining scene directives and world 
descriptions. Scene directives establish parameters such as which integration 
and sampling techniques the renderer must use, the view matrix and other camera 
properties. World descriptions describe the objects and  
materials used to render them. This ensemble of descriptions is called a \textbf{scene}.

\section{Rendering a Scene}
% - the scene
% rendering process
%We know that scenes are, as stated in the previous section, ensembles of descriptions of objects, textures, materials and other important directives. 
%Most physically-based renderers describe scenes using a similar structure, since these descriptions are the first requirement for rendering synthetic images. 
%a 3D scene into a 2D image. 
%
3D scenes are usually created by artists using some modeling 
software (such as Blender \cite{blender}, 3ds Max \cite{3dsmax} or Maya \cite{maya}).
% to draw the objects, choose their materials and then create the object files. 
%These files will then be instanced 
%in the scene file and interpreted by the renderer.
%
% - stating the problem/motivation: making scenes is hard 
% (reference: http://www.laubwerk.com/home/)
But even with an artist’s expertise, creating scenes is still a complex process. 
For instance, scenes created for building overviews and interior design often 
compile hundreds of 3D models and dozens of customized materials and textures, 
as one can see in Figure \ref{fig:intro_complexScene}. Each material and texture 
has to be carefully defined, taking into account the renderer's limitations and 
particularities.
%
Most physically-based renderers describe scenes using a similar structure.

\begin{figure}[h]
  
\includegraphics[width=\textwidth,height=\textheight,keepaspectratio]{images/1_introduction/Laubwerk-Kit-12_Bauclassroom-Exterior.jpg}
  \caption{An example of a complex scene created by Laubwerk Plants Kits, extracted from \cite{laubwerk}}
  \label{fig:intro_complexScene}
\end{figure}

After a scene is created, all the hard work invested by the artist has been, unfortunately, tailored for a specific render. 
%waiting for a possible future use. 
Should the artist choose to renderer the scene using a different rendering system, the scene file so diligently created would have to be 
rewritten or heavily modified.

Converting a scene file from one renderer format to another is a hard and time-consuming task. Unfortunately, the various rendering systems available use proprietary scene description formats. Aside from adapting material and light properties, which can be hard since sometimes renderers do not provide the same features, the 3D object formats supported may not be the same. Manually converting a scene from one format to the other can be extremely time consuming, taking up to several days per scene \cite{tungsten}.


%\section{Improving Physically-Based Rendering}
\section{The Need for Automatic Scene Conversion}

Currently, PBR and MC Ray Tracing are the only practical solution for simulating global illumination effects in complex environments. Due to its high computational cost, an image can take a long time to render - sometimes up to a few days, depending on the complexity of the scene or the technique used. This means that it is practically impossible to use these images in real-time applications.

There are some techniques that aim to improve the time spent rendering PBR images, like improved sampling~\cite{Heck2013, Pilleboue:2015} and reconstruction strategies~\cite{Sen2012, Rousselle2013, Kalantari2015, Bitterli2016}. These techniques are implemented on top of an existing PBR system as a way of using aspects of the MC Ray Tracing algorithm (such as ray-object intersection calculations) that are orthogonal to the proposed methods.

Since converting scenes between renderers is a very time-consuming task, these new techniques are often constrained to demonstrate their results on a limited set of scenes available for the chosen renderer. This apparently simple limitation has profound implications, as it constrains a direct comparison between MC rendering techniques that have been implemented using different rendering systems.

%\section{Project Definition}	

We present {\it a system for automatic conversion among scene file formats used by PBR systems}. Our solution intends to expand the repertoire of scenes available for testing, validation, and benchmarking of PBR algorithms. Currently, our system handles conversions among PBRT v3, Mitsuba, and LuxRender, which are three of the most popular physically-based renderers. 

Extending it to support additional renderers is straightforward. Our solution (discussed in Section~\ref{sec:systemarch}) consists of {\it importing} any source scene description into a canonical representation, which can then be {\it exported} to 
other scene formats. 

\begin{figure*}[h]
  \centering
  \subfloat[PBRT v3]{\includegraphics[width=0.33\linewidth]{images/5_results/coffee/1_from_pbrt.png}
	  \label{breakfast_Lux}	
  }	
  \subfloat[Mitsuba]{\includegraphics[width=0.33\linewidth]{images/5_results/coffee/2_to_mitsuba.png}
	  \label{breakfast_PBRT}		
  }	
  \subfloat[LuxRender]{\includegraphics[width=0.33\linewidth]{images/5_results/coffee/3_to_lux.png}
		  \label{breakfast_Mitsuba}	
  }	
  \caption{Example of automatic scene conversion obtained with our system. \textit{Coffee Maker} rendered with PBRT v3 (left). Rendering produced by Mitsuba (center) and LuxRender (right), using converted scenes (from PBRT v3) for these rendering systems.}
  \label{fig:teaser}
\end{figure*}

Figure~\ref{fig:teaser} illustrates the use of our system to perform automatic conversion of a scene represented in the PBRT v3 format. The images at the center and on the right show, respectively, the renderings produced by Mitsuba and by LuxRender, from converted scenes files. Note the correct representation of the various materials (glass, plastic, and metal).

Our work does not introduce a new physically-based rendering technique per se. Instead, it falls in the area of {\it meta-research} systems, which are systems 
designed to facilitate and improve the research process. Meta-research systems are quite common in computer graphics and computer vision~\cite{MiddleburyStereo, MiddleburyFlow, AlphaMatting, VideoMatting}, where they have led to significant progress in these fields. 
  
Recently, \cite{Santos:2018:FBKSD} introduced a framework for developing and benchmarking MC sampling and denoising algorithms. This is achieved by providing an API that decouples the developed techniques from the used rendering system. 

While it allows a technique to be tested on any rendering system that supports the proposed API, each rendering system is still constrained to a limited set of test scenes. Our system is orthogonal to and complements this API, aspiring to reach full orthogonality among algorithms, rendering systems and scene files.

The {\bf contributions} of our work include:
\begin{itemize}
	\item A system for automatic conversion among scene file formats used by Monte Carlo physically-based rendering systems (Chapter~\ref{sec:systemarch}).
	It enables algorithms implemented on different rendering systems to be tested on similar scene descriptions, giving developers and end user a better 
assessment of the strengths and limitations of MC rendering techniques;
	\item The proposal of a mechanism intending to achieving orthogonality among MC rendering algorithms, rendering systems and scene files (Chapter~\ref{sec:systemarch}). 
This could be achieved when integrated with the API provided in~\cite{Santos:2018:FBKSD}. 
\end{itemize}

\section{Thesis Structure}

This work includes a detailed description of how our solution for converting scenes between renderers was implemented, from system architecture to result images and analysis. Chapter \ref{sec:related_work} discusses previous scene-conversion efforts, and reviews 
similar meta-reseach systems introduced to improve research in computer graphics.
%developed as , as well as other attempts to convert scenes and exporting a generic 
%scene format into different renderer-specific outputs. 
Following the current state-of-the-art, we delve a little into the history of rendering in chapter \ref{sec:theory}, starting with the first algorithms and then moving on to currently used physically-based renderers.
%
Our system is described in chapter \ref{sec:systemarch}, where we explain how our pipeline works and how we converted some renderer-specific directives. Chapter~\ref{sec:???} discusses some results obtained with our system, and  Chapter~\ref{sec:???} presents our conclusions and directions for future exploration.




\chapter{Related Work}
\label{sec:related_work}

In this chapter we discuss relevant, similar systems developed as meta-reseach, such as other attempts to convert scenes and exporting a generic 
scene format into different renderer-specific outputs. This chapter is meant to contextualize the reader in the state-of-the-art in converting PBR scene files.

\section{Scene Conversion}

With the addition of new physically-based renderers to the market, the number of incompatible file formats began to grow. There were a few initiatives to solve this problem, the greatest one being the proposal of a common file format.
Arnaud and Barnes created COLLADA \cite{collada}, a XML schema file format that intended to unify representation of digital assets among 
various graphics software applications. Ever since it became property of the Khronos Group, several companies included a COLLADA module on their 3D modeling 
software or game engines. However, there were few physically-based renderers that adhered to this file format, one of the few being Mitsuba. That might have 
happened because COLLADA files only include information about the geometry present in the scene - they do not store any information about other rendering 
options, such as camera positioning or integration techniques.

To the best of our knowledge, no previous system has performed automatic scene conversion among the major MC rendering systems. Bitterli has converted 32 scenes of various complexities and origins from Tungsten to PBRT v3 and Mitsuba using some scripts~\cite{tungsten}. These scripts, however, are specific for conversions from Tungsten to these two renderers and are not publicly available.

RenderToolbox3 is a MATLAB tool developed for assisting vision research~\cite{rendertoolbox}. It imports a scene containing geometric objects described as COLLADA XML files, and allows one to associate to them reflectance measurements from a MATLAB Psycophysics Toolbox~\cite{Brainard1997}. Such reflectance measurements are converted to multispectral reflectance representations compatible to PBRT and Mitsuba. A script then renders the objects with the associated multispectral representations using PBRT or Mitsuba. RenderToolbox3 is a visualization tool for exploring the impact of different reflectance and illuminating properties on human perception.

The work of \cite{rendertoolbox} is a system that imports COLLADA XML files into a MATLAB system and uses different rendering techniques in order images as close to 
reality as possible. Their system uses integrated PBRT and Mitsuba rendering is capable of storing their data structure into their respective scene file 
formats. The system does not convert scene files among rendering systems. 

\section{Meta-Research Systems in Computer Graphics}

Several systems have been developed to support research in computer graphics. Some well-known examples include Cg~\cite{Mark2003}, Brook~\cite{Buck2004}, and Halide~\cite{Ragan-Kelley2012}.

Cg is a general-purpose programming language designed to support the development of efficient GPU applications, and has stimulated a lot of research efforts in shader-based rendering techniques~\cite{Policarpo2005, Policarpo2006, Wyman2005, Oliveira2007RRT}.  

Brook~\cite{Buck2004} is system for general-purpose computation that allows one to exploit the inherent parallelism of modern GPUs without having to deal with GPU architecture details. These kinds of systems were an inspiration that led to the development of CUDA~\cite{Nickolls2008}.

Halide~\cite{Ragan-Kelley2012} is a system designed to optimize image-processing applications on multiple hardware platforms by separating the algorithm description from its schedule. The system has been recently extended to support differentiable programming for image processing and deep learning~\cite{Li2018-Halide-Diff_Prog}. 

All these systems focus on generating efficient code while freeing the user from GPU architectural details. All goal, in turn, is to make high-quality scenes availability independent of one's choice of rendering system. 

Santos et al. \cite{Santos:2018:FBKSD} have recently presented a framework for developing and benchmarking MC sampling and denoising algorithms. They use an API to decouple algorithms from rendering systems, allowing for the same algorithm to be tested on multiple rendering systems. By doing so, they also increase the set of scenes an algorithm can be tested with. However, in order to use a given test scene, the rendering system for which the scene was created would have to be used as well. 



\section{System Architecture}
Our converting pipeline is subdivided into three main states: the \textbf{Import 
Module}, the \textbf{Canonical Scene Representation} and the \textbf{Conversion 
Module}, as illustrated in Figure \ref{fig:sysarch}. Given an arbitrary input 
file format, our converter is able to import the scene and transform it into a 
generic, canonical representation and then export it to different output 
formats. 

\begin{figure}[h]
\centering
\includegraphics[width=2.5in]{figs/3_system_architecture/architecture.png}
\caption{Illustration of the system pipeline.}
\label{fig:sysarch}
\end{figure}

Our Proof of Concept encompassed \textit{PBRT} \cite{pbrt}, \textit{Mitsuba} 
\cite{mitsuba} and \textit{LuxRender} \cite{luxrender}, as these are three of 
the most popularly used renderers in the community.

\subsection{Import Module}
Most physically-based renderers have a similar way of describing a scene. 
Usually, they divide a scene into two sections: scene-wide rendering options and 
world block. The former defines overall rendering settings (such as which 
rendering or sampling technique should be used) while the latter describes the 
geometry and which materials should be used for rendering.

Our import module specializes in reading and interpreting such scene files. The 
input file is read, parsed and each directive is loaded into our canonical scene 
representation. Since each renderer has its own proprietary file format, we have 
three importing modules: one for each renderer.

\textit{PBRT} and \textit{LuxRender} file formats are composed of structured 
text statements defining all scene directives. Given their structure, a Lex/Yacc 
parser was considered the best choice for these formats. As we intended to keep 
our system in pure Python, we chose to use PLY \cite{ply}, a Python 
implementation of Lex and Yacc.

\textit{Mitsuba}'s file format consists of a XML file. Since there are several 
XML-parsing libraries for Python that can load the hierarchy into a tree data 
structure, we didn't think it necessary to create a Lex/Yacc parser. We chose to 
implement this module using ElementTree \cite{ET}, a XML parsing tool.

\subsection{Canonical Scene Representation}
After loading the scene file, the information obtained from them has to be 
stored somewhere. While most renderers have the same base structure, they differ 
in which parameters can be used to configure the techniques used during the 
rendering process. 

Renderer directives are usually given in the format of a command, followed by a 
type and a list of additional parameters. So, for instance, to specify the path 
integration technique with 8192 samples per pixel in \textit{PBRT} one would 
write the following directive: \textit{Integrator ``path'' ``integer 
pixelsamples'' [8192]}.

In order to establish a common ground for conversion, we 
defined a canonical scene representation. This representation can be easily 
extended incorporate any directives not contemplated in this work.

In this representation, we divide the scene into \textbf{scene-wide rendering 
options} and \textbf{world block}. The rendering options are divided into 
integration technique and sensor options, while the world block is divided into 
lists of shapes, global emitters and material definitions. This structure is 
illustrated in Figure \ref{fig:canonicalrep}.

\begin{figure}[h]
\centering
\includegraphics[width=2.5in]{figs/3_system_architecture/canonicalrep.png}
\caption{Illustration of the canonical scene representation.}
\label{fig:canonicalrep}
\end{figure}

\subsubsection{Scene-wide Rendering Options}
A set of directives specifying the integration and sampling techniques used for 
rendering, camera and film properties. These directives are represented in a 
structure with two fields: a type and a list of parameters.

\subsubsection{World Block}
A set of directives describing the shapes, materials and global emitters present 
in the scene. 

The \textbf{shape} directive is represented in a structure with: a type (cube, 
sphere, ...), an optional area emitter, an optional material reference, an 
optional transformation matrix and a list of parameters. 

The \textbf{material} directive is represented in a structure with: a type, an 
id, an optional texture and a list of parameters.

The \textbf{texture} directive is represented in a structure with: a type, an id 
and a list of parameters.

The \textbf{global emitter} directive is represented in a structure with: a 
type, an optional transformation matrix and a list of parameters.

\subsection{Conversion Module}
Subsection text here.

% An example of a floating figure using the graphicx package.
% Note that \label must occur AFTER (or within) \caption.
% \begin{figure}[!t]
% \centering
% \includegraphics[width=2.5in]{figs/SIBGRAPI2018-banner.png}
% % where an .eps filename suffix will be assumed under latex, 
% % and a .pdf suffix will be assumed for pdflatex; or what has been declared
% % via \DeclareGraphicsExtensions.
% \caption{SIBGRAPI - Conference on Graphics, Patterns and Images.}
% \label{fig_sim}
% \end{figure}

% An example of a double column floating figure using two subfigures.
% (The subfig.sty package must be loaded for this to work.)
% \begin{figure*}[!t]
% \centering
% \subfloat[Case I]{\includegraphics[width=2.5in]{figs/SIBGRAPI2018-banner.png}%
% \label{fig_first_case}}
% \hfil
% \subfloat[Case II]{\includegraphics[width=2.5in]{figs/SIBGRAPI2018-banner.png}%
% \label{fig_second_case}}
% \caption{SIBGRAPI - Conference on Graphics, Patterns and Images.}
% \label{fig_sim2}
% \end{figure*}
%

% An example of a floating table. Note that, for IEEE style tables, the
% \caption command should come BEFORE the table
% \begin{table}[]
% \renewcommand{\arraystretch}{1.3}
% \caption{An Example of a Table}
% \label{table_example}
% \centering
% \begin{tabular}{|c||c|}
% \hline
% One & Two\\
% \hline
% Three & Four\\
% \hline
% \end{tabular}
% \end{table}

\section{Results}
\label{sec:results}

We have tested our system on a large number of scenes, including the 32 scenes available
at Bitterli's rendering resources website~\cite{resources16}. 
%We validated our system's conversion using scenes from Bitterli's 32 resources 
%\cite{resources16}. 
Here, we include a few examples that explore different 
types of materials and bump maps, 3D meshes and primitive shapes, image and 
primitive textures, and area light sources and environment lighting. They include most elements typically found in scenes used by physically-based rendering systems.  The time required to convert a scene is about 0.5 seconds on a typical PC (Intel i5 3.8 GHz).
% 
%We chose scenes that provided a wide variety of directives 
%in order to cover most commonly used directives. These scenes include: different 
%types of materials and bump maps; 3D meshes and primitive shapes; image and 
%primitive textures; area and environment lighting. 
%
The scenes were rendered using Mitsuba 0.5.0, PBRT v3, and LuxRender v1.6 on 
Ubuntu 14.04 LTS. All scenes were rendered using between 5,000 and 8,000 samples per pixel (spp). For any given scene,
the same number of samples per pixel was used with all rendering systems. 

Figure~\ref{fig:teaser} shows a coffee maker containing different materials, including glass, plastic, and metal, as well as textures. The input scene description was provided in the format for PBRT v3, whose rendering is shown on the left. The images at the center and on the right were produced by Mitusuba and LuxRender, respectively, from scene representations automatically converted by our system. 
%from the input scene file. 
Note how the object details have been faithfully preserved in these renderings.

The \textit{Wooden Staircase} scene (Figure~\ref{fig:staircase}) contains many geometric objects and textures. 
A LuxRender scene description was provided as input and its rendering is shown in (a). The images shown in (b) and (c) were produced 
by PBRT v3 and Mitusuba, respectively, from scene representations automatically converted by our system. 

The \textit{Teapot} scene (Figure \ref{fig:teapot}) contains a shiny object, environment lighting, and a procedural texture. The input scene description was provided in the LuxRender format, and its rendering is shown in (a). Figures~\ref{teatpot_PBRT} and \ref{teapot_Mitsuba} show the renderings produced by PBRT v3 and Mitusuba, respectively, from scene descriptions converted by our system.

explores the use of textures and geometric details. This scene was originally modeled for 

We obtained interesting results with \textit{The Wooden Staircase} (Figure 
\ref{fig:staircase}, \textit{Utah Teapot} (Figure \ref{fig:teapot}), 
\textit{Coffee Maker} (Figure \ref{fig:teaser}) and Glass of Water (Figure 
\ref{fig:glass-of-water}). We reached a high degree of visual similarity between 
the original and the converted scenes, which was our goal.

We can observe some difference in coloring between the image rendered by Mitsuba 
(center) and the ones rendered by other two renderers in Figure 
\ref{fig:dining-room}. This particular scene uses a distant environment light to 
emulate the sun. In Mitsuba, this directive is implemented using the technique 
described in \cite{Preetham}, which produces a warm-colored, distant light source. In 
PBRT and LuxRender, this directive is implemented simply as a distant white 
light, which produces the difference shown in the results.

\begin{figure}
\centering

\subfloat[LuxRender]{\includegraphics[width=0.32\linewidth]{figs/4_results/staircase/1_from_lux.png}
	\label{staircase_Lux}
}
\subfloat[PBRT v3]{\includegraphics[width=0.32\linewidth]{figs/4_results/staircase/3_to_pbrt.png}
	\label{staircase_PBRT}
}
\subfloat[Mitsuba]{\includegraphics[width=0.32\linewidth]{figs/4_results/staircase/2_to_mitsuba.png}
	\label{staircase_Mitsuba}
}
\caption{\textit{The Wooden Staircase} scene. Input scene description for LuxRender (a).
Renderings produced by PBRT v3 (b) and Mitsuba (c),
from scene descriptions converted by our system. }
\label{fig:staircase}
\end{figure}

\begin{figure*}
\centering

\subfloat[LuxRender]{\includegraphics[width=0.32\linewidth]{figs/4_results/teapot/1_from_lux.png}
	\label{teapot_Lux}
}
\subfloat[PBRT v3]{\includegraphics[width=0.32\linewidth]{figs/4_results/teapot/3_to_pbrt.png}
	\label{teatpot_PBRT}
}
\subfloat[Mitsuba]{\includegraphics[width=0.32\linewidth]{figs/4_results/teapot/2_to_mitsuba.png}
	\label{teapot_Mitsuba}
}
%
%\includegraphics[width=0.32\linewidth]{figs/4_results/teapot/1_from_lux.png}
%\includegraphics[width=0.32\linewidth]{figs/4_results/teapot/2_to_mitsuba.png}
%\includegraphics[width=0.32\linewidth]{figs/4_results/teapot/3_to_pbrt.png}
\caption{\textit{Teapot} scene. Input scene description for LuxRender (a).	Renderings produced by PBRT v3 (b) and Mitsuba (c),
	from scene descriptions converted by our system.}
\label{fig:teapot}
\end{figure*}

\begin{figure}
\centering
\includegraphics[width=0.325\linewidth]{figs/4_results/veach-bidir/1_from_mitsuba.png}
\includegraphics[width=0.325\linewidth]{figs/4_results/veach-bidir/2_to_pbrt.png}
\includegraphics[width=0.325\linewidth]{figs/4_results/veach-bidir/3_to_lux.png}
%\vspace{-0.2cm}
\subfloat[Mitsuba]{\includegraphics[width=0.325\linewidth]{figs/4_results/cornell-box/1_from_mitsuba.png}
}	
\subfloat[PBRT v3]{\includegraphics[width=0.325\linewidth]{figs/4_results/cornell-box/2_to_pbrt.png}
}	
\subfloat[LuxRender]{\includegraphics[width=0.325\linewidth]{figs/4_results/cornell-box/3_to_lux.png}}
\caption{\textit{Veach, Bidir Room} (top) and \textit{Cornell Box} (bottom). Input scene descriptions for 
 Mitsuba (a). Renderings produced by PBRT v3 (b) and LuxRender (c),
 from scene descriptions converted by our system.}
\label{fig:veach-bidir}
\end{figure}

\begin{figure}
\centering
\subfloat[LuxRender]{\includegraphics[width=0.32\linewidth]{figs/4_results/dining_room/1_from_lux.png}
	\label{breakfast_Lux}	
}	
\subfloat[PBRT v3]{\includegraphics[width=0.32\linewidth]{figs/4_results/dining_room/3_to_pbrt.png}
	\label{breakfast_PBRT}		
}	
\subfloat[Mitsuba]{\includegraphics[width=0.32\linewidth]{figs/4_results/dining_room/2_to_mitsuba.png}
		\label{breakfast_Mitsuba}	
}	
\caption{\textit{The Breakfast Room} scene. Input scene description for LuxRender (a).
	Renderings produced by PBRT v3 (b) and Mitsuba (c),
	from scene descriptions converted by our system. }
\label{fig:dining-room}
\end{figure}

\begin{figure}
\centering
\subfloat[PBRT v3]{\includegraphics[width=0.32\linewidth]{figs/4_results/glass_of_water/1_from_pbrt.png}
	\label{glass_PBRT}		
}	
\subfloat[LuxRender]{\includegraphics[width=0.32\linewidth]{figs/4_results/glass_of_water/2_to_lux.png}
	\label{glass_Lux}		
}	
\subfloat[Mitsuba]{\includegraphics[width=0.32\linewidth]{figs/4_results/glass_of_water/3_to_mitsuba.png}
	\label{glass_Mitsuba}		
}	
\caption{\textit{Glass of Water}. Input scene description for PBRT v3 (a).
	Renderings produced by LuxRender (b) and Mitsuba (c),
	from scene descriptions converted by our system. } 
\label{fig:glass-of-water}
\end{figure}

\subsection{Limitations}
In order to minimize scope issues, we restricted the number of directives 
interpreted by our system. Generally speaking, directives present in only one 
renderer that had no correspondent in the other two renderers were not 
incorporated. That was the case, for instance, for Mitsuba-only materials like 
\textit{phong} or \textit{blendbsdf}. 

We chose not to interpret and convert hair or participating media (volumes, such 
as water or fog) for this PoC. We also did not convert the color for metal 
materials in LuxRender given the issues discussed in \ref{systemarch}. The 
latter can be observed in Figure \ref{fig:veach-bidir} - we can see the lack of a 
copper color on the lamp in the image rendered by LuxRender.






\section{Conclusion}
\label{sec:conclusion}
We presented a system for automatic conversion among scene file formats used by Monte Carlo physically-based rendering systems. 
It enables algorithms implemented using different renderers to be tested on similar scene descriptions, providing better means of assessing the strengths and limitations of MC rendering techniques. 
%
Our system can be easily integrated with the API recently introduced by~\cite{Santos:2018:FBKSD}, allowing researchers and developers to exploit full orthogonality among MC algorithms, rendering systems, and scene files.   

We have demonstrated the effectiveness of our system by converting scene description among three of the most popular MC rendering systems: PBRT v3, Mitsuba, and LuxRender. Providing support to additional renderers only requires specializing the import and export modules described in Section~\ref{sec:systemarch} for the given renderers. Our system is freely available and we encourage developers to provide support for other renderers.  

In the future, we would like to add support for the conversion of hair and participating media, as well as for other rendering systems.
By documenting limitations and incompatibilities found among different renderers, our work might stimulate efforts to reduce these differences.   


% conference papers do not normally have an appendix

% use section* for acknowledgment
%\section*{Acknowledgment}
%
%
%The authors would like to thank...

% trigger a \newpage just before the given reference
% number - used to balance the columns on the last page
% adjust value as needed - may need to be readjusted if
% the document is modified later
%\IEEEtriggeratref{8}
% The "triggered" command can be changed if desired:
%\IEEEtriggercmd{\enlargethispage{-5in}}

\bibliographystyle{IEEEtran}
\bibliography{example}
\end{document}


